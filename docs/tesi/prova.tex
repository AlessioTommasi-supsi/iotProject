\documentclass[12pt,a4paper]{article}

% Pacchetti base
\usepackage[italian]{babel}
\usepackage[utf8]{inputenc}
\usepackage{graphicx}
\usepackage{amsmath}

% Imposta margini della pagina
\usepackage[margin=2.5cm]{geometry}

\title{Documento di Test LaTeX}
\author{Utente}
\date{\today}

\begin{document}

\maketitle

\section{Introduzione}
Questo è un documento di test per verificare il funzionamento del sistema LaTeX.

\section{Formattazione del testo}
Il testo può essere \textbf{in grassetto}, \textit{in corsivo} o \underline{sottolineato}.
Possiamo anche usare combinazioni come \textit{\textbf{grassetto e corsivo}}.

\section{Formule matematiche}
Ecco alcune formule matematiche di esempio:

Inline: $E=mc^2$ è la famosa equazione di Einstein.

A display:
\begin{equation}
    \int_{a}^{b} f(x) \, dx = F(b) - F(a)
\end{equation}

\section{Elenchi}

\subsection{Elenco puntato}
\begin{itemize}
    \item Primo elemento
    \item Secondo elemento
    \item Terzo elemento
\end{itemize}

\subsection{Elenco numerato}
\begin{enumerate}
    \item Primo passo
    \item Secondo passo
    \item Terzo passo
\end{enumerate}

\section{Tabella}
\begin{table}[h]
    \centering
    \begin{tabular}{|c|c|c|}
        \hline
        \textbf{Colonna 1} & \textbf{Colonna 2} & \textbf{Colonna 3} \\
        \hline
        A & 1 & Alpha \\
        B & 2 & Beta \\
        C & 3 & Gamma \\
        \hline
    \end{tabular}
    \caption{Una semplice tabella.}
    \label{tab:esempio}
\end{table}

\section{Conclusione}
Se questo documento viene compilato correttamente, il sistema LaTeX è funzionante.

\end{document}